\documentclass[../main.tex]{subfiles}
\graphicspath{{\subfix{../images/}}}


\begin{document}
\chapter{Introduction}
\par
\tab In principle, quantum computers hold clear promise in addressing problems not generally tractable with classical simulation techniques, and are capable of solving certain exponentially hard problems in polynomial time, taking advantage of quantum phenomena such as superposition and entanglement to achieve parallelism. In practice though, quantum computers still lack in several aspects when compared to their classical counterparts. In contrast to digital computers, where semiconductor technologies have replaced every pre-existing hardware, the case for quantum computers is hardly similar. There are many approaches to constructing a quantum computer. In addition to the numerous underlying technologies that govern their implementation and determine their function, a plethora of different computational architectures also exist. 
\par
In order for quantum computing to be viable, a number of obstacles need to be overcome. Decoherence and systematic errors when performing unitary transformations are the most important problems, since they are limiting the accuracy of the computations. In order to account for these problems, a substantial external active control system to achieve fault-tolerant computing is required. 
\par
In the current report, we will analyze one of the numerous state-of the-art technologies for implementing quantum computers, the Ion Trap Quantum Computer, first devised by Cirac and Zoller \cite{PhysRevLett.74.4091}. We will describe in detail how the ion trapping technology works, as well as discuss different methods of trapping, what types of particles are suitable for this model, as well as how these particles are cooled in order to become isolated from the environment. As an extent, the fundamental operations on ion-trapped qubits are going to be shown, as well as how they are translated to the quantum circuit model. \\

\end{document}